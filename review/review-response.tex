\documentclass{letter}
\usepackage[margin=25mm]{geometry}
\usepackage{csquotes}
\usepackage{siunitx}  % use for all units

% NOTE : Use as \si{\kph} or \si{\mps}
\def\kph{\kilo\meter\per\hour}
\def\mps{\meter\per\second}

\date{January 8, 2026}
\signature{The Authors}

\begin{document}
\begin{letter}{Ergonomics Review Response}
\opening{Dear Reviewers,}

Thank you for the careful review of our work. We revised the paper to address
your comments. The revision is an improved version based on your feedback.

\begin{displayquote}
  \textbf{Editor}: I have now received replies from reviewers regarding your
  manuscript and their comments are appended at the end of this email.  You
  will see that although the reviewers judge there to be some merit in the
  paper, they advise that substantial revisions are required. The reviewers’
  comments appear reasonable and you are invited to consider these and modify
  your paper accordingly.
\end{displayquote}

Our comments are listed below, according to the reviewers' order. Comments from
the reviewers are indented.

\begin{displayquote}
  \textbf{Reviewer 1}: This paper evaluates vibrations with dummies
  representing infants aged 0, 3, and 9 months lying or sitting in five
  strollers and two cargo bicycles with baby seats on 6 road surfaces using the
  ISO 2631 standard. Vibration measurements were conducted using five different
  strollers and two cargo bicycles with two baby seats. In addition to
  reporting the vibration related variables such as RMS acceleration and VDV,
  the authors also conducted health and discomfort assessment. Studying the
  effects of vibrations on children/infants is extremely difficult because
  testing on real individuals is not acceptable.

  The title ``Vibration Characterisation of Strollers and Cargo Bicycles for
  Transporting Infants Including Recommendations for Users, Designers,
  Manufacturers, and Researchers'' seems a little big long with too much
  details. The authors may consider to truncate the ``including
  recommendations….''.
\end{displayquote}

We streamlined the title to ``Vibration Characterisation of Strollers and Cargo
Bicycles for Transporting Infants Over Different Road Surfaces''. We added
``Over Different Road Surfaces'' to make the title clearer to the reader, as
suggested by reviewer 2.

\begin{displayquote}
  It is not clear why the authors included the age 0 (newborn) infants.
  Newborns are either in hospitals or at home. It is unbelievable and is very
  rare that people taking a newborn out using a stroller or cargo bicycles.
\end{displayquote}

We tested newborn sized dummies in lying postures in systems dedicated for their
transport. We are aware that newborns are transported in strollers and cargo
bicycles less frequently compared to older infants. Nonetheless, newborns are
transported by these means in many countries around the world.

One motivation for a study such as ours is to provide information that can
inform guidance on the minimum age for transporting an infant by stroller or
bicycle. Thus, we characterize a encompassing range of infant ages. Testing
different dummy mass and size also quantifies the relationship between body size
and vibration transmission. For these reasons, we measured newborn-sized
dummies.

We added the following in the paper (Section 2.1 Test Equipment) to clarify
this:

``Although newborns are less frequently transported in strollers and cargo
bicycles compared to older infants, these transport means are still used in
daily routines for newborns. For example, in the Netherlands parents transport
newborns by bicycle. Including newborns in the test matrix is therefore a
conservative approach and allows us to consider the worst-case scenario:
newborns have the least developed musculoskeletal system, have the lowest mass,
and are potentially the most vulnerable group.''

\begin{displayquote}
  This paper is much longer than ordinary submission. It is unbelievable that
  even the appendixes have cover 20 pages. The authors should check the word
  count limitations (A typical paper for this journal should be no more than
  8000 words) to determine if the word count is acceptable.
\end{displayquote}

The paper we first submitted contained 8193 words in main article text and was
accepted for review with that disclosure. The appendices are intended to be
electronic supplementary materials and not included in the article word count.
We provide this supplementary material to ensure the reproducibility and
completeness of our work. We have moved the appendices into a separate document
to make this more clear. After the edits made to address this review the word
count of the main article text is now 7374. 

\begin{displayquote}
  Unrelevant issues should be deleted to make the paper concise. An example is
  on P3, Ln 24 ``Cycling is beneficial for health…. leisure.'' There is no need
  to talk about the benefits of bicycling. Another example is the details of
  the dummies, bicycles/strollers, and pavements of the test roads. The
  Appendix E should be deleted, at least. The Effect of Road Surface in A.3 may
  also have little and very limited information.
\end{displayquote}

We agree that the paragraph beginning with ``Cycling is beneficial for
health\ldots'' is unnecessary and have deleted it to make the introduction more
concise. We desire to keep the appendices as supplementary material to ensure
reproducibility and completeness of our work and thus have not removed the
suggested material from the appendices.

\begin{displayquote}
  Paver bricks and tarmac were tested for all vehicles, but only strollers were
  tested on cobblestones, sidewalk pavers, and sidewalk slabs. Even with the
  pavements of the same materials, road surface conditions could be quite
  different (high variations) due to the installation, years of usage, traffic,
  and many other conditions. The road surfaces adopted in this study may
  represent very small samples of all road surfaces.
\end{displayquote}

This comment along with comments from the other reviewers helped us reconsider
the statistical approach we originally proposed. We had made the assumption that
variation in sections of the road surface we traversed during a single trial
could represent independent observations of road surface variability, but you
are correct in that the roads we tested did not offer a large variability
relative to that found in the world for that surface type. We have completely
removed the statistical model and hypothesis tests from the paper because we
could not easily justify that our repetitions constituted independent sampling
from a population of road surfaces. The road surfaces we tested do still allow
for comparisons among them and do give an idea of vibrations induced for similar
road surface geometries. Removal of the statistical analysis does not alter our
conclusions.

Additionally, we added the following sentence to the section ``2.4 Experimental
Protocol'' to address this remark:

``We selected convenient road surfaces representative of commonly occurring
urban conditions in Delft. Although these surfaces cover only part of the full
range of worldwide possibilities, we tested at various locations to capture
local variation in surface quality.''

\begin{displayquote}
  The authors have mentioned that ISO 2631 considers the effects of vibration
  on adults, not on children/infants. Yet, they still perform ISO 2631-1
  weighted vertical RMS acceleration extensively to discuss the effects of the
  vibration on children/infants. The There is no way to validate whether this
  is appropriate or not because there is no ISO vibration standard issued for
  children/infants. The dummies, even with similar size and mass, probably
  don’t have the same biomechanical and phychophysical characteristics of the
  children/infants.
\end{displayquote}

We are aware of the limitations of ISO~2631-1 and we stated this multiple times
in the paper for emphasis. We also recommend that the standards be improved to
address their deficiencies. We decided to report our findings in reference to
the current standard and other references because we want to benchmark against
something, rather than nothing. We believe that sensible conclusions can be made
with reference to the standards, such as our statement: ``We did measure
vibrations that would not be permitted for adults to maintain long-term
occupational health and it is reasonable to believe we should not subject our
infants to the same.'' Our aim in using ISO~2631-1 was not to claim direct
applicability to infants, but rather to provide a well-known framework, widely
used in vibration research. Furthermore, the use of ISO~2631-1 allows us to
compare the results with other research. We now close the introduction with this
addition to further highlight this reason: ``Furthermore, as an example of how
such measurements could be used for assessing discomfort and health effects, we
discuss these in the light of thresholds suggested by ISO~2631-1 and Gao~et.~al
to benchmark against the limited, but established, references.'' We also report
unweighted values for the benefit of future researchers and to highlight the
effect of the weighting.

We do not believe that the findings we report (seat pan vertical acceleration)
are significantly dependent on the biomechanical and physiological
characteristics of the children/infants other than their mass and size. If we
were reporting on other measures (e.g. head motion), then we would need to
further address the deficiencies in the dummies' characteristics as compared to
live humans. We have now noted this in the discussion:

``As is common for adult vibration, accelerations were measured at the
dummy-seat interface rather than on the body. This limits influence of infant
body motion to the measurements.''

\begin{displayquote}
  \textbf{Reviewer 2}: Authors have chosen an interesting topic in the field of
  vibrations and ergonomics especially for infants. Overall, the work is well
  conducted and explained. To make it more interesting to readers, I have some
  suggestions/ comments.

  1. The title looks very general, it should be improved and include the
  targeted age or location from where the data of infants are considered.
\end{displayquote}

We streamlined the title to ``Vibration Characterisation of Strollers and Cargo
Bicycles for Transporting Infants Over Different Road Surfaces''. We added
``Over Different Road Surfaces'' to make the title clearer to the reader.

\begin{displayquote}
2. The abstract starts with "We" that should be avoided.
\end{displayquote}

The abstract now starts with ``The objective of this study was to\ldots''.

\begin{displayquote}
3. Abstract needs revision and include the novelty/ objective/ findings of the
work.
\end{displayquote}

We have edited the abstract to make clear the objective and noted the novelty of
its comprehensiveness. We believe we have already included the primary findings
in the original abstract and the 150 word limit leaves no room to add secondary
findings.

\begin{displayquote}
4. Authors are recommended to avoid "we" in the content.
\end{displayquote}

We thank the reviewer for the suggestion. However, we did not find any
prohibition from Ergonomics on the active voice and it is our preferred style
for scientific writing. The six most recent papers published in Ergonomics as of
December 17, 2025 were all written with the active voice. Thus, we have kept the
active voice as it was in the paper.

\begin{displayquote}
5. In keywords, only ISO is written, but not mentioned the full standard of ISO.
\end{displayquote}

We edited it; now the keyword is: ``ISO~2631''.

\begin{displayquote}
6. The starting paragraph of the introduction can be deleted as it only consists
of the facts that is not related to the technical aspects of the work.
\end{displayquote}

We removed the starting paragraph of the introduction as per your
recommendation.

\begin{displayquote}
7. The introduction can be reduced and should be focused on the previous
literature, novelty not on the mechanism of bicycle or other general topics.
\end{displayquote}

We have removed the first paragraph and the paragraph beginning with ``Cycling
is beneficial for health\ldots'' to address this comment.

\begin{displayquote}
8. The citations are starting with [4] , [20] etc. It should start with [1] [2] etc.
\end{displayquote}

We have now ordered the reference list by the order the citations appear in the
paper.

\begin{displayquote}
9. The materials and methods are well explained.
\end{displayquote}

We thank the reviewer for the compliment.

\begin{displayquote}
10. The results and discussion are also well explained but it shall be better if
authors can compare the results with any previous studies.
\end{displayquote}

We added three more comparisons with previous studies in the section ``4
Discussion''.

\begin{displayquote}
Overall, the work done by authors is appreciable.
\end{displayquote}

We thank the reviewer for the appreciation.

\begin{displayquote}
  \textbf{Reviewer 3}: The manuscript presents a valuable and methodologically
  solid experimental characterization of vibrations transmitted to infants in
  strollers and cargo bicycles, based on extensive field measurements and ISO
  2631-1–based analyses; however, several issues weaken its scientific
  contribution.

  The structure, while detailed, is not always clear: the manuscript is long,
  sometimes repetitive, and sections such as Results, Discussion, and
  Recommendations partially overlap.
\end{displayquote}

We reduced the length of the paper by removing content from the introduction, as
per suggestions from the other two reviewers, and by removing the statistical
analysis. We merged subsections 3.1 and 3.2 to avoid excessive fragmentation of
the paper. We edited the paper for overlap and repetition.

\begin{displayquote}
  The Introduction provides background on whole-body vibration but does not
  clearly articulate the research gap, particularly why the adult-oriented ISO
  2631-1 standard is used despite its known limitations for infants, nor does it
  situate the work adequately within infant-specific vibration literature.
\end{displayquote}

We have more clearly stated the research gap in the opening introduction
paragraph. To explain earlier in the paper why we make use of the ISO~2631-1
standard, we now close the introduction with this addition: ``Furthermore, as
an example of how such measurements could be used for assessing discomfort and
health effects, we discuss these in the light of thresholds suggested by
ISO~2631-1 and Gao~et.~al to benchmark against the limited, but established,
references.`` We believe we situate the paper within the closest literature we
have been able to find and now cite our exhaustive literature study as a
reference for more background information.

\begin{displayquote}
  Although the study cites relevant works (e.g., vibration transmissibility,
  comfort studies, bicycle vibration references), it lacks a deeper comparison
  with previous findings and does not analyze how its results confirm, extend,
  or contradict existing knowledge.
\end{displayquote}

We added a comparison with three previous studies in ``Section 4, Discussion''
noting confirmation and contradiction.

\begin{displayquote}
  The statistical analysis does not include verification of key model
  assumptions. Although ordinary least squares regression is used, the authors
  do not test whether the residuals meet the homoscedasticity requirement.
  Given the clear variability across surfaces, speeds, and vehicle types,
  unequal variance is likely and may affect the reliability of the reported
  significance levels. Including basic diagnostic checks would strengthen the
  robustness of the statistical conclusions (e.g., Levene’s test, Breusch–Pagan
  test, or residual plots)
\end{displayquote}

As mentioned above in response to Reviewer 1, we have decided to drop the
statistical model and analysis from the paper, which obviates addressing this
comment directly. In the process of investigating this remark, we re-examined
another assumption we made for the statistics: whether our observations (each
repetition) were independent. We had convinced ourselves that dividing our long
time series measurements into 20 to 40 second repetitions resulted in
independent observations sampled from a different portion of the traversed road
in a trial. This methodology may be marginally sound, but we no longer feel
confident in this assumption holding due to the more-or-less arbitrary choice in
a 40 second maximum and little variation within a trial's road surface. Choosing
different repetition durations effects the observation distributions and
resulting p-values as well as the heteroscedasticity.

\begin{displayquote}
  The Discussion, while rich in engineering insights, does not critically
  reflect on methodological constraints such as the validity of using dummies
  instead of human infants, the lack of physiological correlates, potential
  bias from specific road surfaces tested in one geographical area, unvalidated
  comfort thresholds.
\end{displayquote}

We added the following in subsection ``4.2 Baby Mass'': ``Due to a lack of
dummies validated for vibration conditions, we tested with dummies of
representative mass and body size. As is common for adult vibration,
accelerations were measured at the dummy-seat interface rather than on the body.
This limits influence of infant body motion to the measurements.'' to emphasize
that seat pan acceleration measurements are not largely effected by the infant
physiology other than mass and size.

Regarding the issue with road surfaces, we edited the section ``2.4 Experimental
Protocol'' by stating as follows: ``We selected arbitrary surfaces
representative of commonly occurring urban conditions in Delft, the Netherlands.
Although these surfaces cover only part of the full range of possibilities, we
travelled various distances to capture variation in surface quality.'' We have
also removed the statistical analysis which inherently implied no bias in road
surface choice.

We stressed the limitations of the ISO thresholds applicability and
methodological constraints of our study throughout the entire section ``4
Discussion''. See the next comments for more explanation of this.

\begin{displayquote}
  The key limitation concerns the applicability of the ISO 2631-1 standard used
  throughout the analysis. This standard was developed for ``healthy adults''
  in occupational and transportation contexts, which implicitly excludes
  infants and young children, whose anatomy, vibration sensitivity, postural
  constraints, and developmental vulnerability differ substantially from those
  of adults. As a result, applying adult-derived thresholds to infant transport
  systems may lead to misinterpretation of comfort or health risk levels. While
  the authors acknowledge this to some extent, a deeper reflection on the
  implications of using a non-validated standard for this population would
  strengthen the methodological transparency and contextual accuracy of the
  study.
\end{displayquote}

In Section 5.1 we state: ``We did measure vibrations that would not be permitted
for adults to maintain long-term occupational health and it is reasonable to
believe we should not subject our infants to the same.'' We do not have any
plausible explanations pointing towards infants being able to tolerate larger
magnitude vibration than an adult, so we feel confident in making this
statement. Our hypothesis is that any future validated vibration limits for
infants would be lower than those of adults. If use of the generally accepted
adult ISO limits for infants results in infants no longer being vibrated above
these limits, it is hard to imagine a resulting negative health or comfort
outcome for the infant in that case. If someone were to treat the adult ISO
limits in an inverse perspective, in that it is completely acceptable to vibrate
an infant at the limits but never over, then such an interpretation may very
likely lead to negative health and comfort results for the infant. We state
caution as clearly as we can and multiple times to dissuade such an
interpretation but believe that providing some limits are better than none.

We have updated the summary in 5.1 with this wording to expose our deeper
reflection: ``We did measure vibrations that would not be permitted for adults
to maintain long-term occupational health and it is reasonable to believe we
should not subject our infants to the same. The preceding statement should
certainly not be interpreted to mean that vibration lower than these limits is
deemed acceptable for infants. The relative vulnerability of infants points more
towards infant limits being lower than those of adults. Further research to
establish direct evidence to infant health and comfort would possibly permit
more or less caution than we conclude below in our recommendations.''

\begin{displayquote}
  Overall, the non-applicability of the ISO 2631-1 standard to infants and
  young children substantially limits the strength and impact of the
  conclusions. Given these limitations, the results are ultimately valid only
  in a comparative sense between the different stroller and cargo-bike models
  tested.
\end{displayquote}

We do agree that the lack of any validated standards for infant vibration limits
the strength of what we can conclude, but we have done our best to explain the,
albeit limited, applicability and utility of the ISO~2631 standard in this
context and why we feel confident in the reasonableness of our presented
conclusions and recommendations. We do not agree that the results are valid only
in a comparative sense between the different stroller and cargo bicycle models.
That is one virtue of our approach and it can be used by designers to make
relative improvements, but our study also presents absolute raw and ISO-weighted
measures of vertical vibration at the seat pan that can be compared to similar
measures in other vehicles or used to replicate these vibrations in simulations.

\closing{Sincerely,}
\end{letter}
\end{document}