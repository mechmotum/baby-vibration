\documentclass{letter}
\usepackage[margin=25mm]{geometry}
\usepackage{csquotes}

\date{17 December 2025}
\title{Ergonomics Review Response}

\begin{document}
Dear Reviewers,

Thank you for the careful review of our work. We have revised the paper to
address your comments both for clarification of various methods and details as
well as providing background and support for some of our choices. The revision
is an improved version based on your feedback.

\begin{displayquote}
  \textbf{Editor}: I have now received replies from reviewers regarding your
  manuscript and their comments are appended at the end of this email.  You
  will see that although the reviewers judge there to be some merit in the
  paper, they advise that substantial revisions are required.  The reviewers’
  comments appear reasonable and you are invited to consider these and modify
  your paper accordingly.
\end{displayquote}

TODO

\begin{displayquote}
  \textbf{Reviewer 1}: This paper evaluates vibrations with dummies
  representing infants aged 0, 3, and 9 months lying or sitting in five
  strollers and two cargo bicycles with baby seats on 6 road surfaces using the
  ISO 2631 standard. Vibration measurements were conducted using five different
  strollers and two cargo bicycles with two baby seats. In addition to
  reporting the vibration related variables such as RMS acceleration and VDV,
  the authors also conducted health and discomfort assessment. Studying the
  effects of vibrations on children/infants is extremely difficult because
  testing on real individuals is not acceptable.

  The title ``Vibration Characterisation of Strollers and Cargo Bicycles for
  Transporting Infants Including Recommendations for Users, Designers,
  Manufacturers, and Researchers'' seems a little big long with too much
  details. The authors may consider to truncate the ``including
  recommendations….''.
\end{displayquote}

TODO

\begin{displayquote}
  It is not clear why the authors included the age 0 (newborn) infants.
  Newborns are either in hospitals or at home. It is unbelievable and is very
  rare that people taking a newborn out using a stroller or cargo bicycles.
\end{displayquote}

We are aware that newborns are less frequently carried outdoors compared to
older infants. However, in daily routine, strollers are used for short trips
(medical appointments, brief walks, going with older siblings to school,
sports, and so on), particularly in urban environments. Parents also transport
newborns by bicycle in The Netherlands and also likely in other countries.

An extremely common question parents have is what the minimum age for
transporting an infant by stroller or bicycle is (as shown by the abudance of
internet articles). We want to characterize the full range of infant ages so
that this question can be answered definitiely in the future.

Including newborns allowed us to consider the worst-case scenario: newborns
have the least developed musculoskeletal system, making them potentially the
most vulnerable group. It is a conservative approach of our study, that does
not exclude tests with older infants (9 months).

For these reasons, we measured newborn sized dummies.

TODO : Add a short statement in the paper that states this?

\begin{displayquote}
  This paper is much longer than ordinary submission. It is unbelievable that
  even the appendixes have cover 20 pages. The authors should check the word
  count limitations (A typical paper for this journal should be no more than
  8000 words) to determine if the word count is acceptable.
\end{displayquote}

The paper has 8193 words, which we disclosed to the editor at submission. The
appendices are intended to be electronic supplementary materials and not word
counted. We provide this supplementary material to ensure the reproducibility
and completeness of our work.

TODO : Separate appendices into a new PDF and name it supplementary materials.

\begin{displayquote}
  Unrelevant issues should be deleted to make the paper concise. An example is
  on P3, Ln 24 ``Cycling is beneficial for health…. leisure.'' There is no need
  to talk about the benefits of bicycling. Another example is the details of
  the dummies, bicycles/strollers, and pavements of the test roads. The
  Appendix E should be deleted, at least. The Effect of Road Surface in A.3 may
  also have little and very limited information.
\end{displayquote}

We desire to keep the appendices as electronic supplementary material to ensure
reproducibilty and completeness of our work.

\begin{displayquote}
  Paver bricks and tarmac were tested for all vehicles, but only strollers were
  tested on cobblestones, sidewalk pavers, and sidewalk slabs. Even with the
  pavements of the same materials, road surface conditions could be quite
  different (high variations) due to the installation, years of usage, traffic,
  and many other conditions.  The road surfaces adopted in this study may
  represent very small samples of all road surfaces.
\end{displayquote}

Yes, the road surfaces we selected do not represent a large quality and
maintenance variation of that surface. We choose surfaces that were at typical
maintenance levels observed The Netherlands. We traveled various lengths, at
least kilometer in the case of the bicycles,  on these surfaces to encounter
some variation in surface quality.

TODO: add a sentence in the paper that states "these surfaces do not represent
large variation in road surface quality"

\begin{displayquote}
  The authors have mentioned that ISO 2631 considers the effects of vibration
  on adults, not on children/infants. Yet, they still perform ISO 2631-1
  weighted vertical RMS acceleration extensively to discuss the effects of the
  vibration on children/infants. The There is no way to validate whether this
  is appropriate or not because there is no ISO vibration standard issued for
  children/infants. The dummies, even with similar size and mass, probably
  don’t have the same biomechanical and phychophysical characteristics of the
  children/infants.
\end{displayquote}

We have stated these limitations in the study, some more than once. We also
recommend that the standards be improved to address its deficiencies. We decide
to report our findings in reference to the current standard and other
references because we want to benchmark against something rather than nothing.
We believe that conservative conclusions can be made via the standards such as
``We did measure vibrations that would not be permitted for adults to maintain
long-term occupational health, and it is reasonable to believe we should not
subject our infants to the same.'' Our aim in using ISO 2631-1 was not to claim
direct applicability to infants, but rather to provide a well-known framework,
widely used in vibration research.

\begin{displayquote}
  \textbf{Reviewer 2}: Authors have chosen an interesting topic in the field of
  vibrations and ergonomics especially for infants. Overall, the work is well
  conducted and explained. To make it more interesting to readers, I have some
  suggestions/ comments.

  1. The title looks very general, it should be improved and include the targeted age or location from where the data of infants are considered.
\end{displayquote}

TODO

\begin{displayquote}
2. The abstract starts with "We" that should be avoided.
\end{displayquote}

Active voice is not prohibited by Ergonomics and is our preferred writing style
for scientific writing. The six most recent papers published in Ergonimics
checked on December 17, 2025 were all written with active voice. Thus, we have
kept active voice as it was in the paper.

\begin{displayquote}
3. Abstract needs revision and include the novelty/ objective/ findings of the
work.
\end{displayquote}

TODO

\begin{displayquote}
4. Authors are recommended to avoid "we" in the content.
\end{displayquote}

Active voice is not prohibited by Ergonomics and is our preferred writing style
for scientific writing. The six most recent papers published in Ergonimics
checked on December 17, 2025 were all written with active voice. Thus, we have
kept active voice as it was in the paper.

\begin{displayquote}
5. In keywords, only ISO is written, but not mentioned the full standard of ISO.
\end{displayquote}

TODO

\begin{displayquote}
6. The starting paragraph of the introduction can be deleted as it only consists of the facts that is not related to the technical aspects of the work.
\end{displayquote}

TODO

\begin{displayquote}
7. The introduction can be reduced and should be focused on the previous literature, novelty not on the mechanism of bicycle or other general topics.
\end{displayquote}

TODO

\begin{displayquote}
8. The citations are starting with [4] , [20] etc. It should start with [1] [2] etc.
\end{displayquote}

TODO

\begin{displayquote}
9. The materials and methods are well explained.
\end{displayquote}

Thank you for the compliment.

\begin{displayquote}
10. The results and discussion are also well explained but it shall be better
if authors can compare the results with any previous studies.
\end{displayquote}

TODO: add some comparisons to values from prior studies in the dicussion

\begin{displayquote}
Overall, the work done by authors is appreciable.
\end{displayquote}

Thank you, we are happy you appreciate the work.

\begin{displayquote}
  \textbf{Reviewer 3}: The manuscript presents a valuable and methodologically
  solid experimental characterization of vibrations transmitted to infants in
  strollers and cargo bicycles, based on extensive field measurements and ISO
  2631-1–based analyses; however, several issues weaken its scientific
  contribution.

  The structure, while detailed, is not always clear: the manuscript is long,
  sometimes repetitive, and sections such as Results, Discussion, and
  Recommendations partially overlap.
\end{displayquote}

\begin{displayquote}
  The Introduction provides background on
  whole-body vibration but does not clearly articulate the research gap,
  particularly why the adult-oriented ISO 2631-1 standard is used despite its
  known limitations for infants, nor does it situate the work adequately within
  infant-specific vibration literature.
\end{displayquote}

TODO

\begin{displayquote}
  Although the study cites relevant works (e.g., vibration transmissibility,
  comfort studies, bicycle vibration references), it lacks a deeper comparison
  with previous findings and does not analyze how its results confirm, extend,
  or contradict existing knowledge.
\end{displayquote}

TODO

\begin{displayquote}
  The statistical analysis does not include verification of key model
  assumptions. Although ordinary least squares regression is used, the authors
  do not test whether the residuals meet the homoscedasticity requirement.
  Given the clear variability across surfaces, speeds, and vehicle types,
  unequal variance is likely and may affect the reliability of the reported
  significance levels. Including basic diagnostic checks would strengthen the
  robustness of the statistical conclusions (e.g., Levene’s test, Breusch–Pagan
  test, or residual plots)
\end{displayquote}

The residuals of both our models are normally distributed (non-significant
Omnibus \(\ki^2\) and Jarque-Bera tests). The Durbin-Watson test indicates no
autocorrelation. 

\begin{displayquote}
  The Discussion, while rich in engineering insights, does not critically
  reflect on methodological constraints such as the validity of using dummies
  instead of human infants, the lack of physiological correlates, potential
  bias from specific road surfaces tested in one geographical area, unvalidated
  comfort thresholds.
\end{displayquote}

TODO

\begin{displayquote}
  The key limitation concerns the applicability of the ISO 2631-1 standard used
  throughout the analysis. This standard was developed for ``healthy adults''
  in occupational and transportation contexts, which implicitly excludes
  infants and young children, whose anatomy, vibration sensitivity, postural
  constraints, and developmental vulnerability differ substantially from those
  of adults. As a result, applying adult-derived thresholds to infant transport
  systems may lead to misinterpretation of comfort or health risk levels. While
  the authors acknowledge this to some extent, a deeper reflection on the
  implications of using a non-validated standard for this population would
  strengthen the methodological transparency and contextual accuracy of the
  study.
\end{displayquote}

TODO

\begin{displayquote}
  Overall, the non-applicability of the ISO 2631-1 standard to infants and
  young children substantially limits the strength and impact of the
  conclusions. Given these limitations, the results are ultimately valid only
  in a comparative sense between the different stroller and cargo-bike models
  tested.
\end{displayquote}

Increase in vibration intensity undoutedly correlates to larger internal forces
in the vibrated object. It should be undisputed, in general, that adults can
endure larger forces than infants for equivalent injury or discomfort. This
then implies that health and comfort thresholds for infants are lower than
those for adults. So stating that we shouldn't subject infants to vibrations
known to cause discomfort and health issues to adults should not be
controversial and referencing the standards for adults at least provides an
extreme limit that if exceeded is most certainly not good for the infant. This
standard is the only substatinal benchmarkable reference we can use to put the
values we measure into some context. We believe that providing this context is
better than providing no context because infants are also made of flesh and
bones as adults are. We hope our work prompts future researchers to seek the
thresholds for infants so these thresholds are not exceeded if they are lower
than the adult thresholds. We believe that placing our results relative to some
known, is preferrable providing them only in relative comparison among the
tested scenarios. We have stated our conclusions and recommendations ``in spite
of the standard's limitations`` based on this reasoning.

\closing{Sincerely, \\
The Authors}
\end{document}
