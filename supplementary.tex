% Style Guide
% - use British English, exceptions: stroller vs buggy or pram, sidewalk vs pavement
% - active voice is preferred over passive voice, minimize passive
% - use past tense to describe what we did
% - try to avoid subsub sections (keep to two layers)
% - use siunitx for all units, surround with brackets [\si{\meter}], note the
%   common definitions, e.g. \si{\kph}
% - use tilde ~ to prevent line breaks in words that should stay together, e.g.
%   something~\cite{} or ISO~2631-1
% - only use "significant" when referring to statistical significance
% - use "infant", not "baby"
% - use "bicycle" instead of "bike"
% - use "amplitude spectrum/spectra" for what we plot
% - use "standard" not "norm"
% - use the oxford comma
% - TODO: acceleration vs accelerations?
% - Data breakdown terms are "Session", "Trial" (of specific "Scenario"),
%   "Repetition"
% -  Referencing: Figure~\ref{fig:name}, Equation~\ref{eq:name},
%    Section~\ref{sec:name}, Appendix~\ref{app:name}, Table~\ref{tab:name}
% - inline urls \href{https://www.e.com}{my hyperlink}
% - do not use [h] for figures except in appendix, let latex place them where it
%   wants
% - figures should be 300 dpi and sized to the actual physical dimension in the
%   paper
% Other Notes
% - be careful about editing autogenerated tables, they will be overwritten

\documentclass[a4paper]{article}

\usepackage{amsmath}  % extra math features
\usepackage{authblk}
\usepackage{booktabs}  % nice tables
\usepackage[margin=25mm]{geometry}
\usepackage{graphicx} % required for inserting images
\usepackage{hyperref} % clickable hyperlinks
\usepackage{multirow}  % pandas to_latex() uses this
\usepackage{siunitx}  % use for all units
\usepackage{subcaption}  % for subfigures

% hyperref settings for better looking links
\hypersetup{
  colorlinks=true,
  linkcolor=blue,
  urlcolor=blue,
  citecolor=blue,
}

% NOTE : Use as \si{\kph} or \si{\mps}
\def\kph{\kilo\meter\per\hour}
\def\mps{\meter\per\second}

\title{
  DRAFT: v10 \\
  Supplementary Materials for:\\
  Vibration Characterisation of Strollers and Cargo Bicycles\\
  for Transporting Infants Over Different Road Surfaces
}

\author[1]{Gabriele Dell'Orto}
\author[2]{Brecht Daams}
\author[1]{Riender Happee}
\author[1]{Georgios Papaioannou}
\author[1]{Arjo~J.~Loeve}
\author[1]{Jesper Meijerink}
\author[1]{Thomas Valk}
\author[1]{Jason K. Moore\footnote{corresponding author: j.k.moore@tudelft.nl,
+31 (0)15 278 3556}}
\affil[1]{Delft University of Technology, Delft, The Netherlands}
\affil[2]{Daams Ergonomie, Laren, The Netherlands}

\begin{document}

\maketitle

\color{red}
\section*{\centering Notice}
\begin{center}
  This paper has been submitted for peer review and is subject to change with
  revisions.
\end{center}
\color{black}

\tableofcontents

\newpage

% Ensure subsections retain "A.1", "A.2", etc.
%\addcontentsline{toc}{section}{Appendix} % Optional: Add to TOC
%\renewcommand{\thesection}{A}
%\renewcommand{\thesubsection}{\thesection.\arabic{subsection}}

\section{Additional Exploratory Statistics}
\subsection{Effect of Speed}
%
Figure~\ref{fig:compare-bicycle-speed} shows the variation in ISO~2631-1
weighted vertical RMS acceleration across the tested speeds. On paver
bricks, the RMS acceleration increased by approximately 1.6\(\times\) when increasing
the speed from \SIrange{12}{25}{\kilo\meter\per\hour}.The effect of speed while cycling on tarmac was less drastic, but still showed a slight increase.
%
\begin{figure}
  \centering
  \includegraphics[width=160mm]{fig/SeatBotacc_ver-bicycle-speed-compare.png}
  \caption{Seat pan ISO 2631-1 weighted vertical acceleration RMS versus speed
  for all cargo bicycle repetitions. Shaded regions represent the 95\%
  confidence intervals from a simple linear regression that ignores
  \(x_\textrm{Cargo Bicycle}\). ALT TEXT: Cargo bicycle vertical acceleration
  versus speed for each road surface.}
  \label{fig:compare-bicycle-speed}
\end{figure}

\subsection{Effect of Dummy Size}
%
Figure~\ref{fig:compare-baby-mass} shows the ISO~2631-1 weighted vertical RMS
acceleration for each repetition for all vehicles, and compares the dummy sizes.
Substantial variation has been shown to relate significantly to the surface and
within cargo bicycles, as well as to speed.
%There were larger accelerations in the bicycle (high speeds) versus the stroller (low speeds), likely mostly attributed to the different testing speeds. 
When comparing the vertical RMS acceleration values between cargo bicycles and
strollers, there are no obvious differences due to baby size, i.e. each dummy
size experienced a similar range of acceleration magnitudes when the vehicle type
and the road surface are ignored. For bicycles, the lightest dummy sometimes
experienced a higher acceleration than the heavier dummy, but high and low
accelerations were observed across the tested speed range.
%
\begin{figure}
  \centering
  \includegraphics[width=160mm]{fig/SeatBotacc_ver-baby-mass-compare.png}
  \caption{Seat pan ISO~2631-1 weighted vertical RMS acceleration grouped by
  baby age (and thus size \& mass) for all repetitions with colour indicating
  the mean speed of the trial. The lightest colour dots are strollers (4 km/h) and the
  remaining are cargo bicycles (8-24 km/h). ALT TEXT: Vertical acceleration for
  each repetition grouped by dummy age representation.}
  \label{fig:compare-baby-mass}
\end{figure}

\subsection{Effect of Road Surface}
%
Figure~\ref{fig:compare-road-surface} shows ISO~2631-1 weighted vertical RMS
acceleration from repetitions grouped into the various road surfaces tested. All
vehicles were tested on paver bricks and tarmac, but only strollers were tested
on cobblestones, sidewalk pavers, and sidewalk slabs, i.e. light colour dots
(strollers are present in each column).
It is notable that tarmac almost always induced lower RMS acceleration than
other road types regardless of speed and vehicle type.
The sidewalk slabs and cobblestones have very similar acceleration ranges for
all strollers.
Paver bricks and sidewalk pavers appear to have a similar range of RMS
acceleration for the same 5~\si{\kilo\meter\per\hour} walking speeds.
Paver bricks cause relatively large accelerations at high travel speeds in
cargo bicycles. Paver bricks result in approximately 4-5\(\times\) higher accelerations compared to tarmac.
%
\begin{figure}
  \centering
  \includegraphics[width=160mm]{fig/SeatBotacc_ver-road-surface-compare.png}
  \caption{Seat pan ISO 2631-1 weighted vertical RMS acceleration grouped by
  road surface with colour indicating the mean speed of the repetition. The
  lightest colour dots are strollers (4 km/h), and the rest are cargo bicycles
  (8-24 km/h). ALT TEXT: Vertical acceleration for each repetition grouped by
  road surface.}
  \label{fig:compare-road-surface}
\end{figure}

\newpage
\section{Shock Tests}
\label{app:shock}

We identified the maximum peak acceleration values for each trial
(Equation~\ref{eq:max-acc}), then averaged them across repetitions to obtain the
results listed in Table~\ref{tab:shock} for different vehicles and
configurations. There is a significant variation between tests, with the peaks for
the bicycles sometimes reaching the full scale of the accelerometer (\(\pm\)16~g).
The seat pan accelerations for strollers are generally lower than
those experienced with bicycles, but in strollers the configuration (0 or 9 months) may play
a large role. Among bicycles, the Melia seems to transmit
lower accelerations compared to Pebble, 3 months, for both the Keiler and the Urban Arrow. Strollers with the baby seat configuration for 9
months show much lower acceleration compared to the baby cot for 0 months
(Bugaboo and Maxi-Cosi). Surprisingly, the vintage strollers (Old Rusty and Green
Machine) performed very well during the shock test, resulting in the lowest seat
pan acceleration among all the vehicles tested.
Figure~\ref{fig:shock_vehicle_comparison} shows the peaks of the vertical
acceleration recorded at the seat pan, grouped by different vehicles, for the
shock test. As noted in Table~\ref{tab:shock}, the vintage strollers Old Rusty and
Green Machine show lower acceleration values. We do not clearly distinguish any
trend with speed for the bicycles (Keiler and Urban Arrow). During tests
involving Keiler and strollers, we cannot exclude that the front wheels
(left and right) hit the bump at slightly different time instants.   

Regarding the
shock test, the analysis presented in Appendix~\ref{app:shock} was conducted on
unfiltered data (not downsampled). We selected the maximum (absolute) peak from
the time history of the vertical acceleration of the seat pan starting from the
events' time histories (Figure \ref{fig:shock_time-history}), limiting it to
16~g when the peak exceeded that maximum.

\begin{align}
  \textrm{MAX}_{a_{z}} =\max|a_{z}(t_n)|
  \label{eq:max-acc}
\end{align}

%
\begin{figure}
  \centering
  \includegraphics[width=160mm]{fig/session015-t5-shock-bicycle-keiler-pebble-0-SeatBotacc_ver-rep0.png}
  \caption{Raw seat pan vertical acceleration versus time from session 015:
  Keiler tricycle during the shock test. ALT TEXT: Vertical acceleration versus
  time during an example shock test.}
  \label{fig:shock_time-history}
\end{figure}

\begin{figure}
  \centering
  \includegraphics[width=160mm]{fig/SeatBotacc_ver-shock-test-compare.png}
  \caption{Vertical acceleration recorded at the seat pan during the shock test
  per each trial, grouped by vehicles. The colour indicates the mean speed of
  the trial. The lighter the colour the lower the speed. ALT TEXT: Peak vertical
  acceleration during shock tests grouped by vehicle.}
  \label{fig:shock_vehicle_comparison}
\end{figure}

\begin{table}
  \centering
  \caption{Mean of the maximum seat pan acceleration across trials in
  \si{\mps\squared} recorded for shock test, for different vehicles and baby
  masses.}
  \label{tab:shock}
  \footnotesize
  \begin{tabular}{lllrrrr}
    \toprule
     & &  & Target Speed & Trial Count & Max Acceleration \\
    Vehicle Type & Model & Seat, Baby & [km/h] & & [m/s²] \\
    \midrule
    Strollers & Bugaboo & Cot, 0 mo  & 5 & 3 & 146 \\
              &         & Seat, 9 mo & 5 & 4 & 49 \\
    \cline{2-6}
              & Green Machine & Cot, 0 mo & 5 & 4 & 17 \\
    \cline{2-6}
              & Maxi-Cosi & Cot, 0 mo  & 5 & 2 & 131 \\
              &           & Seat, 9 mo & 5 & 4 & 35 \\
    \cline{2-6}
              & Old Rusty & Seat, 9 mo & 5 & 4 & 31 \\
    \cline{2-6}
              & Stokke & Cot, 0 mo  & 5 & 4 & 35 \\
              &                     & Seat, 9 mo & 5 & 4 & 51 \\
    \cline{1-6}
    Bicycles & Keiler & Melia, 3 mo & 5 & 2 & 52 \\

             &               & Melia, 3 mo & 12 & 2 & 85 \\

             &               & Melia, 3 mo & 20 & 2 & 124 \\
    \cline{2-6}
             &               & Pebble, 0 mo & 5 & 2 & 113 \\

             &               & Pebble, 0 mo & 12 & 2 & 140 \\

             &               & Pebble, 0 mo & 20 & 2 & 115 \\
    \cline{2-6}
             &               & Pebble, 3 mo & 5 & 2 & 151 \\

             &               & Pebble, 3 mo & 12 & 2 & 160 \\

             &               & Pebble, 3 mo & 20 & 2 & 145 \\
    \cline{2-6}
             & Urban Arrow & Melia, 3 mo & 5 & 2 & 43 \\

             &               & Melia, 3 mo & 12 & 2 & 43 \\

             &               & Melia, 3 mo & 25 & 2 & 42 \\
    \cline{2-6}
             &               & Pebble, 0 mo & 5 & 1 & 50 \\
    
             &               & Pebble, 0 mo & 12 & 2 & 63 \\
    
             &               & Pebble, 0 mo & 25 & 2 & 130 \\
    \cline{2-6}
             &               & Pebble, 3 mo & 5 & 2 & 156 \\
    
             &               & Pebble, 3 mo & 12 & 2 & 160 \\
    
             &               & Pebble, 3 mo & 25 & 2 & 160 \\
    \bottomrule
  \end{tabular}
\end{table}


\section{Future Work}
% 
There are four possible directions for future work: more in-depth analysis of
the present measurements, more research on the vibrations transmitted by infant
transport systems, more research on the effects of vibration on infants, and
development of a benchmark.

\paragraph{Concerning further analysis of the present measurements:}
%
We acquired data from four other sensors on the vehicles, each with three
accelerometer and three gyroscope time histories, for a total of 30 time
histories of possible interest. This paper provides a look into the experiments
via four metrics: ISO weighted vertical RMS acceleration, maximum acceleration,
peak frequency, and bandwidth of the seat pan sensor. The collected data can
also be used to investigate the transmissibility from sensor to sensor, as well
as rotational vibration effects. Investigating these further can give a more
complete picture of the connections to health and comfort. This work also gives
a benchmark against which new designs can be tested to show reduction in
vibration.

\paragraph{Concerning more research on vibrations generated by infant transport
systems:}
%
Some products, scenarios, and variables have not yet been investigated for their
effect on vibration. For example, running with an infant in a jogging stroller is a subject of interest as vibrations are likely to exceed health limits. Furthermore, recumbent postures in
jogging strollers are estimated to increase the vibration load on the head of the
infant. It is important to establish the actual vibration load on the head in
practice.

\paragraph{Concerning more research on the effects of vibration on infants:}
%
The frequency weighting of the ISO~2631-1 standard is not designed or validated
to characterise health and comfort for infants or children, for short durations,
and for non-erect seating. Research is urgently needed to develop a new standard
with a more appropriate frequency weighting to improve assessment of the comfort
and health effects of whole-body vibration of infants and children, also for
short durations and for non-erect seating. Furthermore, tests with more
realistic dummies and/or real infants are necessary to investigate how the
infant itself moves when excited by various vibrations in different postures.
The results will contribute to the assessment of the transmissibility of
vibration in children, which is needed to deduce the vibrations transmitted to
the head. 

\paragraph{Concerning the development of a benchmark:}
%
At this moment, no requirements exist for the vibrational properties of child
transport systems in the standards for strollers, cargo bicycles, bicycle seats,
bicycle trailers, and car seats. Due to the magnitude of vibrations that can
occur during child transport, it is imperative to include requirements for the
vibrational properties in the standards for these products. At this moment, it
is difficult to develop new requirements because there is not enough
information. With the information resulting from more research on infant
vibration, a benchmark for the vibration properties of infant transportation
products needs to be developed. Tests as performed in this study mark a first
milestone. Road surfaces, speeds, dummies, posture, and metrics could be
standardised. Over the years, these could be refined building upon scientific
research. Acceptance thresholds could be set to accept current products that
perform well.  This will enable the inclusion of (minimal) requirements for
vibration properties in the standards, which would greatly increase the health
and comfort of infants who must use these products. 

\section{Experimental Equipment}
\label{app:equipment}

We considered the use of ``crash test dummies'' (also known as anthropomorphic
test devices or ATDs), which are designed and commonly used to test child seats
in car crash tests. However, the closest available body sizes for crash test
dummies do not meet the body sizes desired for this investigation. Furthermore, crash test dummies are of a fundamentally different design, based on scaling of adult biomechanical data rather than child data and with dynamic properties designed for crash conditions, not for vibration testing. Therefore, crash test dummies were not used for the envisaged vibration tests. 
 The following lists the dummies used:
%
\begin{description}
  \item[Dummy 0 months] weight 3.48~\si{\kg}, size 50~\si{\cm}. Dollkit 20'',
  ``Andi Asleep'' (product code AW380008).
  (\href{http://www.atelier-wiesje.nl/index.php?item=9912---dollkit-20-_-andi-asleep--_-armen-_-_-benen----available&action=article&group_id=10000164&aid=2163&lang=NL}{Webpage})
  
  \item[Dummy 3 months] weight 5.90~\si{\kg}, size 62.5~\si{\cm}. Dollkit 25'',
  ``Asia - Limited Edition'' (product code 300287).
  (\href{http://www.atelier-wiesje.nl/index.php?action=article&aid=2556&group_id=10000176&lang=NL}{Webpage})
  \item[Dummy 9 months] weight 8.90~\si{\kg}, size 70~\si{\cm}. Dollkit 28'',
  ``Hailey'' (product code 304137).
  (\href{http://www.atelier-wiesje.nl/index.php?action=article&aid=1974&group_id=10000133&lang=NL}{Webpage})
\end{description}
%
The strollers and bicycles used for the experiments are shown below, along with their wheelbase, wheel diameter, sensor location, and orientation for all the tested vehicles.

\begin{description}
\item[Strollers] A selection of vintage and modern strollers for carrying
infants. The modern strollers are configurable as cots for newborns and as seats
for older infants.
%
\begin{description}
  \item[Bugaboo Fox 5] Featuring large wheels and a suspension system that makes
  it an all-terrain stroller, according to the manufacturer. It has large
  puncture-resistant tyres and an air-permeable mattress.
  (\href{https://www.bugaboo.com/nl-nl/speciale-aanbiedingen/bugaboo-fox-5-bassinet-and-seat-stroller-black-base-midnight-black-fabrics-midnight-black-sun-canopy-PV006272.html?gad_source=1&gclid=Cj0KCQiA1Km7BhC9ARIsAFZfEIuB-6fQPRl5KyIHVJtion5lyD_Z1Qn-IP3shWoB_HXFozM1ySTXNfgaAgzFEALw_wcB&gclsrc=aw.ds}{Bugaboo
  website})
  \item[Green Machine] An unknown brand vintage perambulator with a cot, dating
  approximately from the 1970s that includes a leather strap suspended cot in a
  metal frame. The wheels are relatively large compared to modern strollers.
  \item[Maxi-Cosi Street Plus] This stroller comes with a cot configuration (0-6
  months) and an infant seat option. It has large wheels with only a marginal suspension system.
  (\href{https://www.maxi-cosi.nl/kinderwagens/street-plus?color_swatch_id=5519}{Maxi-Cosi website})
  \item[Old Rusty] An unknown brand vintage seat-style stroller dating approximately from the 1970s that includes a metal spring suspension system and wheels that are slightly larger than those on modern strollers.
  \item[Stokke BABYZEN YOYO 0+] Among the smallest foldable strollers on the market. It has only a marginal suspension system and smaller wheels compared to other strollers available on the market.
  (\href{https://tinylibrary.nl/products/kinderwagen-babyzen-yoyo-0-plus}{Model
  rented for the experiment})
\end{description}
\item[Cargo Bicycles] Two modern Dutch ``bakfietsen'' in which different baby
seats can be mounted.
%
\begin{description}
  \item[Keiler Tricycle] A tadpole (two wheels in the front, one in the rear)
  cargo tricycle without electric assist. This vehicle will not roll into
  curves. However, differing road surface unevenness at the left and right front
  wheels will induce lateral roll and lateral acceleration in infants. We added
  masses in specific locations to simulate the presence of an electric motor and
  battery package, in order to compare the results with those of the Urban
  Arrow. The tricycle was equipped with tyres CST XPEDIUM Safe 47-559
  (Cheng Shin Tire, Yuanlin, Taiwan) on the rear, and Schwalbe Green Comfort Road Cruiser
  K-Guard 3 47-507 (Ralf Bohle GmbH, Reichshof-Wehnrath, Germany) on
  the front. We set both tyres to an inflation pressure of 300~\si{\kPa}.
  (\href{https://kashop.nl/product/keiler-bakfiets/}{Product Webpage}) 
  \item[Urban Arrow Cargo Bicycle] An electric cargo bicycle that is popular in
  the Netherlands. This vehicle has two inline wheels, which makes it roll into
  curves like a regular bicycle. It is featured by a long frame that can carry
  loads (e.g., a child in a car seat or a baby shell) placed in between the
  rider and the front wheel. We used the ''Family Performance Plus'' model, with
  an electric drivetrain from Bosch Performance Line (65~\si{\N}\si{\m},
  250~\si{\W}). It was equipped with tyres Schwalbe Big Ben Plus 55-559 on the
  rear, and Schwalbe Big Ben Plus 55-406 on the front (Ralf Bohle GmbH, Reichshof-Wehnrath, Germany). We set both tyres to an
  inflation pressure of 300~\si{\kPa}, according to the range recommended by the
  manufacturer.
  (\href{https://urban arrow.com/}{Urban Arrow website})
\end{description}
\item[Baby Seats] These two baby seats were mounted in the cargo bays of the two
cargo bicycles.
%
\begin{description}
  \item[Melia Baby Shell] Baby seat specifically designed for cargo bicycles
  meant for babies from 2 to 9 months old. Mounted to the seat platform in both
  cargo bicycles.
    (\href{https://melia.nl/product/babyschaal-4-seizoenen-comfort/}{Melia
    Webpage})
  \item[Maxi-Cosi X Joolz Pebble Pro i-Size Car Seat] Can be adapted to be
  mounted on cargo bicycles with a fitting fixing system (e.g. ``Urban Arrow
  Baby Car Seat Adapter''). Comes with an integrated suspension system. The same
  adapter was used in both cargo bicycles.
    (\href{https://www.maxi-cosi.nl/autostoelen/pebble-pro-i-size}{Maxi-Cosi
    Webpage})
\end{description}
\end{description}

Additional sensor locations are as follows.

\begin{enumerate}
  \item \textbf{Front Wheel} This IMU was mounted on the front wheel fork (the
  caster wheel for the stroller). This was the non-rotating measurement point
  closest to the ground. The purpose of this sensor was to measure the road
  roughness filtered out only by the tyres.  
  \item \textbf{Frame} On the cargo bicycle, an IMU was placed below the frontal
  cargo bay and clamped to the frame. This was to provide an understanding of
  the damping characteristics of the bicycle frame, together with an estimation
  of the effect of the suspension system. On the stroller, the IMU was taped
  under the baby seat.
  \item \textbf{Seat Head} IMU taped into the baby seat (on the soft mattress),
  directly in contact with the dummy's head. This IMU measured the vibration
  transmitted to the head contact point.
\end{enumerate}

\begin{figure}
  \centering
  \subcaptionbox{IMU on the front fork}{\includegraphics[width=0.45\textwidth, angle=-90]{fig/FW_UA.jpg}}
  \subcaptionbox{IMU under the cargo bay}{\includegraphics[width=0.45\textwidth, angle=-90]{fig/BT_UA.jpg}}
  \caption{IMU locations on the Urban Arrow. The white 3D printed
  sensor supports are visible in (a) and (c). The last figure (d) shows how the
  sensors were taped to the seat pan and backrest. ALT TEXT: Photos of extra IMU
  mounting locations.}
  \label{fig:sensors_UA2}
\end{figure}

\begin{figure}[htbp]    % use [htbp] to place the figures where I prefer
  \centering
  \subcaptionbox{Lateral view}{\includegraphics[width=75mm]{fig/TechDraw_UA-Joolz_lat.pdf}}
  \subcaptionbox{Front view}{\includegraphics[width=80mm]{fig/TechDraw_UA-Joolz_front.pdf}}
  \caption{IMU locations on the Urban Arrow, equipped with
  Pebble. ALT TEXT: Photos of the Urban Arrow with Pebble seat.}
  \label{fig:tech_drawing_UA_Joolz}
\end{figure}

\begin{figure}[htbp]    % use [htbp] to place the figures where I prefer
  \centering
  \subcaptionbox{Lateral view}{\includegraphics[width=75mm]{fig/TechDraw_UA-Melia_lat.pdf}}
  \subcaptionbox{Front view}{\includegraphics[width=80mm]{fig/TechDraw_UA-Melia_front.pdf}}
  \caption{IMU locations on the Urban Arrow, equipped with Melia. ALT TEXT:
  Photos of the Urban Arrow with Melia seat.}
  \label{fig:tech_drawing_UA_Melia}
\end{figure}

\clearpage
\begin{figure}[htbp]
  \centering
  \subcaptionbox{Lateral view}{\includegraphics[width=75mm]{fig/TechDraw_Trike-Joolz_lat.pdf}}
  \subcaptionbox{Front view}{\includegraphics[width=80mm]{fig/TechDraw_Trike-Joolz_front.pdf}}
  \caption{IMU locations on the Keiler, equipped with
  Pebble. ALT TEXT: Photos of the Keiler with the Pebble seat.}
  \label{fig:tech_drawing_Trike_Joolz}
\end{figure}


\begin{figure}[htbp]
  \centering
  \subcaptionbox{Lateral view}{\includegraphics[width=75mm]{fig/TechDraw_Trike-Melia_lat.pdf}}
  \subcaptionbox{Front view}{\includegraphics[width=75mm]{fig/TechDraw_Trike-Melia_front.pdf}}
  \caption{IMU locations on the Keiler, equipped with
  Melia. ALT TEXT: Photos of the Keiler with the Melia seat.}
  \label{fig:tech_drawing_Trike_Melia}
\end{figure}

\clearpage
\begin{figure}[htbp]
  \centering
  \subcaptionbox{Lateral view}{\includegraphics[width=70mm]{fig/TechDraw_Bugaboo0_lat.pdf}}
  \subcaptionbox{Front view}{\includegraphics[width=70mm]{fig/TechDraw_Bugaboo0_front.pdf}}
  \caption{IMU locations on the Bugaboo, configured for a 0-month-old baby. ALT
  TEXT: Photos of the Bugaboo with cot.}
  \label{fig:tech_drawing_Bugaboo0}
\end{figure}

\begin{figure}[htbp]
  \centering
  \subcaptionbox{Lateral view}{\includegraphics[width=70mm, angle=-90]{fig/TechDraw_Bugaboo9_lat.pdf}}
  \subcaptionbox{Front view}{\includegraphics[width=70mm]{fig/TechDraw_Bugaboo9_front.pdf}}
  \caption{IMU locations on the Bugaboo, configured for a 9-month-old baby. ALT
  TEXT: Photos of the Bugaboo with seat.}
  \label{fig:tech_drawing_Bugaboo9}
\end{figure}

\clearpage
\begin{figure}[htbp]
  \centering
  \subcaptionbox{Lateral view}{\includegraphics[width=70mm, angle=-90]{fig/TechDraw_Maxicosi0_lat.pdf}}
  \subcaptionbox{Front view}{\includegraphics[width=70mm]{fig/TechDraw_Maxicosi0_front.pdf}}
  \caption{IMU locations on the Maxi-Cosi, configured for a 0-month-old baby.
  ALT TEXT: Photos of the Maxi-Cosi with cot.}
  \label{fig:tech_drawing_Maxicosi0}
\end{figure}

\begin{figure}[htbp]
  \centering
  \subcaptionbox{Lateral view}{\includegraphics[width=70mm, angle=-90]{fig/TechDraw_Maxicosi9_lat.pdf}}
  \subcaptionbox{Front view}{\includegraphics[width=70mm,angle=-90]{fig/TechDraw_Maxicosi9_front.pdf}}
  \caption{IMU locations on the Maxi-Cosi, configured for a 9-month-old baby.
  ALT TEXT: Photos of the Maxi-Cosi with seat.}
  \label{fig:tech_drawing_Maxicosi9}
\end{figure}

\clearpage
\begin{figure}[htbp]
  \centering
  \subcaptionbox{Lateral view}{\includegraphics[width=70mm, angle=-90]{fig/TechDraw_YOYO0_lat.pdf}}
  \subcaptionbox{Front view}{\includegraphics[width=70mm]{fig/TechDraw_YOYO0_front.pdf}}
  \caption{IMU locations on the Stokke, configured for a 0-month-old baby. ALT
  TEXT: Photos of the Stokke with cot.}
  \label{fig:tech_drawing_YOYO0}
\end{figure}

\begin{figure}[htbp]
  \centering
  \subcaptionbox{Lateral view}{\includegraphics[width=70mm, angle=-90]{fig/TechDraw_YOYO9_lat.pdf}}
  \subcaptionbox{Front view}{\includegraphics[width=70mm]{fig/TechDraw_YOYO9_front.pdf}}
  \caption{IMU locations on the Stokke, configured for a 9-month-old baby. ALT
  TEXT: Photos of the Stokke with seat.}
  \label{fig:tech_drawing_YOYO9}
\end{figure}

\clearpage
\begin{figure}[htbp]
  \centering
  \subcaptionbox{Lateral view}{\includegraphics[width=70mm, angle=-90]{fig/TechDraw_GreenM0_lat.pdf}}
  \subcaptionbox{Front view}{\includegraphics[width=70mm]{fig/TechDraw_GreenM0_front.pdf}}
  \caption{IMU locations on the Green Machine, configured for a 0-month-old
  baby. ALT TEXT: Photos of the Green Machine.}
  \label{fig:tech_drawing_GreenM0}
\end{figure}

\begin{figure}[htbp]
  \centering
  \subcaptionbox{Lateral view}{\includegraphics[width=70mm, angle=-90]{fig/TechDraw_OldR9_lat.pdf}}
  \subcaptionbox{Front view}{\includegraphics[width=70mm]{fig/TechDraw_OldR9_front.pdf}}
  \caption{IMU locations on the Old Rusty, configured for a 9-month-old baby.
  ALT TEXT: Photos of the Old Rusty.}
  \label{fig:tech_drawing_OldR9}
\end{figure}

\clearpage
\newpage

\section{Location and Pictures of the Experiment Areas}
\label{app:location}
%
\begin{itemize}
  \item \textbf{Strollers} were tested with dummies of 0 months and 9 months at
  5~\si{\kph} at:
  \begin{itemize}
      \item Tarmac
      \item Paver bricks % Klinkers
      \item Sidewalk pavers % Stoeptegels
      \item Cobblestones 
      \item Sidewalk slabs (concrete blocks with gaps in between) % Aula blocks
      \item Shock bump: A 30x30~\si{\mm} square section aluminium bar 
  \end{itemize}
  \item \textbf{Cargo bicycles} were tested with dummies of 0 months and 3
  months. Tests were performed at 12~\si{\kph} for both vehicles. The Urban Arrow
  was also tested at 25~\si{\kph} whereas the Keiler was tested at
  20~\si{\km\per\hour} for safety reasons (due to wobbling). The test surfaces
  are:
  \begin{itemize}
      \item Tarmac
      \item Paver bricks
      \item Shock bump: A 30~\si{\mm} x 30~\si{\mm} square section aluminium bar (also tested at 5~\si{\kph})
      \end{itemize}
  \item \textbf{Baby seats} Both the Pebble and the
  Melia was tested on each cargo bicycle using the same mounting
  systems.
\end{itemize}

\subsection{For Bicycles}

\begin{itemize}
    \item Tarmac
\begin{figure}[htbp]
  \centering
  {\includegraphics[width=70mm, angle=-90]{fig/Tarmac_bicycle.pdf}}
  \caption{Tarmac surface where we tested bicycles. ALT TEXT: Photo of tarmac
  for cargo bicycles.}
  \label{fig:tarmac_bicycle}
\end{figure}

The bicycle experiment on the tarmac was conducted along Leeghwaterstraat, 2628 CA Delft, The Netherlands (GPS coordinates: 52.001053, 4.369071).

    \item Paver bricks
\begin{figure}[htbp]
  \centering
 {\includegraphics[width=70mm, angle=-90]{fig/Paver_bricks_bicycle.pdf}}
  \caption{Paver bricks surface where we tested bicycles. ALT TEXT: Photo of paver bricks for cargo bicycles.}
  \label{fig:paver_bricks_bicycle}
\end{figure}

The bicycle experiment on the paver bricks was conducted along Hertog Govertkade and Kanaalweg, 2611 DD Delft, The Netherlands (GPS coordinates: 52.006264, 4.363013).\\
Details of paver brick: rectangular shape, dimensions 195x95~\si{\mm} (gap in between: 7±2~\si{\mm}).

\clearpage

    \item Shock
\begin{figure}[htbp]
  \centering
  \includegraphics[width=70mm, angle=0]{fig/Shock_bicycle.pdf}
  \caption{We performed a shock test with bicycles riding over a 30x30~\si{\mm}
  square section bar. ALT TEXT: Photo of cargo bicycle shock test.}
  \label{fig:shock_bicycle}
\end{figure}

The bicycle shock experiment was conducted along Leeghwaterstraat, 2628 CA Delft, The Netherlands (same location of the experiment on the tarmac, GPS coordinates: 52.001053, 4.369071).\\
Details of the shock experiment: we rode over a square-sectioned aluminium bar 30x30~\si{\mm}.

\end{itemize}

\subsection{For Strollers}

\begin{itemize}
    \item Tarmac

\begin{figure}[htbp]
  \centering
    \includegraphics[width=70mm, angle=-90]{fig/Tarmac_stroller.pdf}
  \caption{Tarmac test area. ALT TEXT: Photo of tarmac for strollers.}
  \label{fig:tarmac_stroller}
\end{figure}

The bicycle experiment on the tarmac was conducted along Julianalaan, 2628 BG Delft, The Netherlands (GPS coordinates: 52.002727, 4.366845).

\clearpage

    \item Cobblestone

\begin{figure}[htbp]
  \centering
  \includegraphics[width=60mm, angle=-90]{fig/Cobblestone_stroller.pdf}
  \caption{Cobblestone surface where we tested strollers. ALT TEXT: Photo of
  cobblestone for strollers.}
  \label{fig:cobblestone_area_stroller}
\end{figure}

The bicycle experiment on the cobblestone was conducted along Julianalaan, 2628 BG Delft, The Netherlands (GPS coordinates: 52.002727, 4.366845).
Details of cobblestone: rectangular shape, dimensions 180x125~\si{\mm} (gap in between: 20±4~\si{\mm}).

    \item Sidewalk pavers

\begin{figure}[htbp]
  \centering
    \includegraphics[width=60mm, angle=-90]{fig/Sidewalk_pavers_stroller.pdf}
  \caption{Sidewalk pavers test area for strollers. ALT TEXT: Photo of sidewalk
  pavers for strollers.}
  \label{fig:sidewalkpavers_area_stroller}
\end{figure}

The experiment was conducted along Prins Bernhardlaan, 2628 CN Delft, The Netherlands (same location as Paver bricks test, GPS coordinates: 52.003147, 4.369530).\\
Details of sidewalk bricks: rectangular shape, dimensions 290x290~\si{\mm} (gap in between: 12±2~\si{\mm}).

\clearpage

    \item Sidewalk slabs

\begin{figure}[htbp]
  \centering
    \includegraphics[width=70mm, angle=-90]{fig/Sidewalk_slabs_stroller.pdf}
  \caption{Sidewalk slab test area where we conducted the test with strollers.
  ALT TEXT: Photo of sidewalk slabs for strollers.}
  \label{fig:sidewalkslabs_area_stroller}
\end{figure}

The experiment was conducted in front of TU Delft Aula Conference Centre (Building 20), Mekelweg 5, 2628 CC Delft, The Netherlands (GPS coordinates: 52.002250, 4.372665).\\
Details of sidewalk slab: made of concrete, rectangular shape, dimensions 2000x990~\si{\mm} (gap in between: 160±3~\si{\mm}).

    \item Paver bricks
\begin{figure}[htbp]
  \centering
  \includegraphics[width=50mm, angle=-90]{fig/Paver_bricks_stroller.pdf}
  \caption{Details of the paver bricks test area. ALT TEXT: Photo of paver
  bricks for strollers.}
  \label{fig:paverbrick_area_stroller}
\end{figure}

The experiment was conducted along Prins Bernhardlaan, 2628 CN Delft, The Netherlands (same location as Sidewalk paver test, GPS coordinates: 52.003147, 4.369530).\\
Details of paver brick: rectangular shape, dimensions 195x95~\si{\mm} (gap in between: 7±2~\si{\mm}).

\clearpage

    \item Shock
\begin{figure}[htbp]
  \centering
  \includegraphics[width=60mm, angle=-90]{fig/Shock_stroller.pdf}
  \caption{Area where we tested strollers during the shock experiment. ALT TEXT:
  Photo of shock test for strollers.}
  \label{fig:shock_area_stroller}
\end{figure}

The experiment was conducted inside TU Delft Mechanical Engineering faculty (Building 34 - Ground floor, aisle in front of Gezelschap Leeghwater office), Mekelweg 2, 2628 CD Delft, The Netherlands (GPS coordinates: 52.000587, 4.372224).\\
Details of the shock experiment: we pushed the stroller over a square section aluminium bar 30x30~\si{\mm}.

\end{itemize}

\end{document}